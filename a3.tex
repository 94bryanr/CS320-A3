\documentclass{article}
\author{Bryan Reilly}
\title{CS320: Assignment 3}

% Do not indent paragraphs
\setlength\parindent{0pt}

% 1 inch margins
\usepackage{fullpage}

% For source code
\usepackage{minted}

\begin{document}
\maketitle{}

\subsection*{Problem 1}
My solution finds the transitive closure of any graph represented by an adjacency matrix.
I iterate through each node $u$,
find what nodes are reachable from $u$ using depth first search,
then use that information to build a transitive closure.

This is the Python code:

\inputminted{python}{q1.py}

The running time of this code is $O( |N| * (|E|+|N|) )$.
I've come to this via the following argument.

The outer loop of the $transitiveClosure$ function will run $|N|$ times.
Within that loop we call $DFS$, which is $O(|E|)$ since at worst it will run through each edge in the graph.
In the same loop we then update the adjacency graph, which happens in $O(|N|)$ time because at worst it must update $|N|-1$ entries in the matrix.
Thus the inside of our loop runs in $O(|E| + |N|)$ time, and the outer loop runs $O(|N|)$ times.
This gives us a final bound of $O( |N| * (|E|+|N|) )$.


\subsection*{Problem 2}
Since each node has a degree of at least 2, there are no leaf nodes
(a leaf node is by definition a node of degree 1).
This means that it will be possible to find a cycle without the need to backtrack
(we can not hit a dead end).
By using a depth-first search and checking for twice explored nodes, we can find a cycle in $O(|V|)$ time.
The running time is $O(|V|)$ because every time we travel down an edge we ``remove'' a node from our remaining search pool of nodes.

Here is the algorithm:

\inputminted{python}{q2.py}

\subsection*{Problem 3}
We may show a strongly connected graph also contains any two nodes $(u,v)$ in a cycle
via the following argument.
Assume G is strongly connected, then there exists a directed path from $u$ to $v$,
and there exists a directed path from $v$ to $u$.
A cycle is defined a a subset of the edge set of G that forms a path such that the first node of the path corresponds to the last.
Thus, we may take our path from $u$ to $v$ as the first part of our cycle, and the path from $v$ to $u$ as the second part of our cycle.
We know both of these paths exist because the graph is strongly connected.
Since we start at $u$ and go through $v$, then come back to $u$, we have a cycle that contains $u$ and $v$.
Thus, if a graph is strongly connected, any two nodes are contained in a cycle.

\subsection*{Problem 4}
\subsubsection*{Adjacency Matrix}
An adjacency matrix implementation allows us to check in $O(1)$ time whether or not a given edge exists in a graph.
The graph must take $|V|^2$ space since we have $\Omega (|V|^2)$ edges.
It takes $\Omega (|V|)$ time to check all incident edges of a node, even if there are less than $|V|$ incident edges.
\subsubsection*{Adjacency List}
An adjacency list requires only $O(|E| + |V|)$ space since we need $|V|$ arrays,
where the length of all the arrays is at most $O(|E|)$.
Since we have $\Omega (|V|^2)$ edges, we will still require $O(|V|^2)$ space,
and thus we save no space over an adjacency matrix representation.
The time it takes to find edge $e$ is proportional to the lowest degree'd node that $e$ is attatched to.
Once a node is found, its neighbors can be found in constant time per neighbor.
\subsubsection*{Preference}
Since the graph has $\Omega (|V|^2)$ edges,
both implementations will take the same amount of space.
It will take $O(|V|)$ time to find all the neighbors of a node in both implementations.
The adjacency matrix will be able to check if an edge $(u,v)$ exists in the graph in $O(1)$ time,
while the adjacency list does the same operation in proportion to the degree $u$ and $v$.
Thus, the adjacency matrix will be the more suitable representation for this type of graph,
since it will have faster edge checking.


\end{document}
