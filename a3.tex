\documentclass{article}
\author{Bryan Reilly}
\title{CS320: Assignment 3}

% Do not indent paragraphs
\setlength\parindent{0pt}

% 1 inch margins
\usepackage{fullpage}

% For source code
\usepackage{minted}

% Custom colors
\usepackage{color}
\definecolor{deepblue}{rgb}{0,0,0.5}
\definecolor{deepred}{rgb}{0.6,0,0}
\definecolor{deepgreen}{rgb}{0,0.5,0}

\begin{document}
\maketitle{}

\subsection*{Problem 1}
My solution finds the transitive closure of any graph represented by an adjacency matrix.
I iterate through each node $u$,
find what nodes are reachable from $u$ using depth first search,
then use that information to build a transitive closure.

This is the Python code:

\inputminted{python}{q1.py}

The running time of this code is $O( |N| * (|E|+|N|) )$. We come to this via the following argument.

The outer loop of the $transitiveClosure$ function will run $|N|$ times.
Within that loop we call $DFS$, which is $O(|E|)$, on the current node.
In the same loop we then update the adjacency graph, which happens in $O(|N|)$ time.
Thus the inside of our loop runs in $O(|E| + |N|)$ time, and the outer loop runs $O(|N|)$ times.
This gives us a final bound of $O( |N| * (|E|+|N|) )$.


\subsection*{Problem 2}
Since each node has a degree of at least 2, there are no leaf nodes.
This means that it will be possible to find a cycle without the need to backtrack.
By simply using a depth-first search and checking for run-overs we can find a cycle in $O(|V|)$ time.

Here is the algorithm:

\inputminted{python}{q2.py}

EXPLAIN WHY NOT O|N| time!!!!! Each edge ``removes'' a node

\subsection*{Problem 3}
We may show a strongly connected graph also contains any two nodes $(u,v)$ in a cycle
via the following argument.
Assume G is strongly connected, then there exists a directed path from $u$ to $v$,
and there exists a directed path from $v$ to $u$.
A cycle is defined a a subset of the edge set of G that forms a path such that the first node of the path corresponds to the last.
Thus, we may take our path from $u$ to $v$ as the first part of our cycle, and the path from $v$ to $u$ as the second part of our cycle.
We know both of these paths exist because the graph is strongly connected.
Since we start at $u$ and go through $v$, then come back to $u$, we have a cycle that contains $u$ and $v$.
Thus, if a graph is strongly connected, any two nodes are contained in a cycle.

\subsection*{Problem 4}


\end{document}
